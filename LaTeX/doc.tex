\documentclass[12pt, a4paper, draft]{report}
\usepackage{palatino}
\usepackage{parskip}

\begin{document}

% -- Title --
\title{An Exploration of Cryptography}
\author{Internation School of Helsinki\\\\
    Siddharth Mahendraker\\
    \texttt{siddharth\_mahen@me.com}\\\\
    Word Count: ~820}
\date{2012}
\maketitle

% -- ToC --
\setcounter{page}{1}
\pagenumbering{roman}
\tableofcontents
\clearpage

% -- The Goal --
\section{The Goal}
\setcounter{page}{1}
\pagenumbering{arabic}

%Question: "What ingenious ideas and processes are involed in
%modern cryptographic theory, and how do they function in practice?"

I have always been interested in computer sciences. It has captivated me
since my youth, and to this day, I continue to write programs, publish
open source software, and contribute to others' work.

Therefore, when the time came to pick a topic for my personal project,
computer sciences seemed like the obvious choice.

My almost insatisfiable craving for computer science related knowledge
has led me down several very interesting paths. I have previously taken my
own time to learn about topic such as fundamental computation theory,
artificial intelligence, algorithm and data structure theory and software
engineering. However, computer science still holds many wonders for
me to discover, and cryptography was one of them.

I chose to learn about cryptography because it was a field of computer
science I genuinely knew nothing about, yet contained so many intersting
ideas. What really drew me in was the fundamental problem that cryptography
solved: it allowed people to communicate securly over completely insecure
channels. I was really interested in learning about the mathematical theory
underneath cryptography and how it could be implemented in the real world.

%I think the area of interaction to which cryptography most relates is
%human ingenuity. This is because cryptography (and for that matter any
%field in computer science) is built by taking abstract ideas and inventing
%creative solution the problems in the field. In cryptography, these solutions
%take the form of alorithms and their corresponding pratical implementations.

I think the area of interaction to which cryptography most relates is
human ingenuity. This is because cryptography (and for that matter any
field in computer science) is based on inventing creative, concrete
solutions to abstract problems. In cryptography, these solutions
take the form of cryptographic alorithms and their implementations
and these problemes take the form of mathematical equations.

Based on my area of interaction and my interests, human ingenuity and
learning about both applied and theoretical cryptography, I came up with the
following inquiry question: "What ingenious ideas and processes are involved
in modern cryptographic theory, and how do they function in practice?".

I decided that the best way to achieve my goal and answer my inquiry
question was to write three seperate cryptographic algorithms embodying
my research and then compare and constrast these algorithms in a research
report. This goal is particularly well suited to my needs as it balances
both the applied and the theoretical parts of cryptography, allowing me to
focus on the entire subject as a whole.

% What specifications did you put in place to help you successfully
% complete your investigation and your project overall?

% -- Selection of Sources --
\section{Selection of Sources}

My research spanned a wide variety of subjects in both theoretical and
practical cryptography.

One of the most useful books I have used is Applied Cryptography
by Bruce Schneier\footnotemark. It gave me invaluable information into the
practical application of cryptography with solid definitions,
great bredth and full code examples. I was sure this book was a good
resource because the author, Bruce Schneier is a very well known
security technologist, who currently serves on the board of directors
of the International Association of Cryptologic Research and is a member
of the Advisory Board for the Electronic Privacy Information Center while
working at a consulting firm specializing in cryptography and computer
security. Furthermore, he has also designed one of the strongest block
ciphers currently available, called \texttt{bcrypt}. I found this book
by asking about good introductory texts to cryptography on Quora.

The next invaluable resource I used is An Introduction to Mathematical
Cryptography by Joseph H. Silverman et al\footnotemark. This book gave me
a complete and through understanding of the underlying mathematics behind
a wide variety of crpytographic algorithms, as well as insight into
intersting mathematical techniques which could be used to break these
algorithms. The book was initial an extremly tough read, as I was not used
to it's rigorous proof based method. However, with practice and much time
spent on the wonderful examples at the end of each chapter, I was able to
use this book effectively. All of the authors of this book are proffessors
of mathematics are Brown University, and have all published numerous
research articles in their respective fields of mathematics, therefore I
beleive that this book contains trustworthy information.

\footnotetext[1]{
Applied Cryptography - Protocols, Algorithms and Source Code in C
by Bruce Schneier
}

\footnotetext[2]{
An Introduction to Mathematical Cryptography
by Jeffery Hoffstein, Jill Pipher and Joseph H. Silverman
}

Another huge resource which I have used regularly is Khan Academy. It's
a sort of online learning resource which teaches you subjects in small
YouTube clips in less than 10 minutes. This was particularly useful if I
wanted to review something such as matrix multiplication, if I hadn't done
it in a while. Just to brush up my math skills in general.

Aside from these resources, I have also referred to a many different
individual articles from various online sources, ranging from an article on
elliptic curve cryptography form the NSA to articles from Wolfram Alpha
about the applications of cellular automata in cryptography.

When it came to picking my resources, I opted for quality over quantity.
Although I did not chose an enourmous selection of resources as some
students have, the resources I have chosen are of excellent quality and
are ultra dense in information and each entry in my bibliography has
been thoroughly vetted.

\section{Application of Information}

\section{Achieving the Goal}

\section{Reflecting on Learning}

\end{document}
