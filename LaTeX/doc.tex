\documentclass[12pt, a4paper, draft]{report}
\usepackage{palatino}
\usepackage{parskip}
\usepackage{amsmath}
\usepackage[titles]{tocloft}

\begin{document}

% -- Bookeeping
\renewcommand*\thesection{\arabic{section}}
\renewcommand{\cftsecfont}{\bfseries}

\setlength\cftbeforesecskip{5pt}
%\setlength\cftaftertoctitleskip{2pt}

% -- Title --
\title{An Exploration of Cryptography}
\author{Internation School of Helsinki\\\\
    Siddharth Mahendraker\\
    \texttt{siddharth\_mahen@me.com}\\\\
    Word Count: 2500}
\date{2012}
\maketitle

% -- ToC --
\setcounter{page}{1}
\pagenumbering{roman}
\tableofcontents
\clearpage

% -- The Goal --
\section*{The Goal}
\addcontentsline{toc}{section}{The Goal}
\setcounter{page}{1}
\pagenumbering{arabic}

%Question: "What ingenious ideas and processes are involed in
%modern cryptographic theory, and how do they function in practice?"

The realm of computer science has captivated me since my youth. It has
always been an interest of mine which I have persued inside and outside
of school. These days, I continue to write programs in my spare time,
publish open source software, and contribute to other open source
projects over the Internet.

Therefore, when the time came to pick a topic for my personal project,
a topic in the field of computer science seemed like an obvious choice.

Through my own interest, I had already taken time to learn about many
of the interesting subtopics in computer science, such as computational
complexity theory, artificial intelligence, algorithm and data structure
design and software engineering. As such, I wanted to try something
a little different; something new and challeging.

I chose to learn about cryptography because it was a field of computer
science I genuinely knew nothing about, yet contained so many intersting
ideas. What really drew me in was the fundamental problem that cryptography
solved: it allowed people to communicate securly over completely insecure
channels. I was really interested in learning about the mathematical theory
underneath cryptography and how it could be implemented in the real world.

%I think the area of interaction to which cryptography most relates is
%human ingenuity. This is because cryptography (and for that matter any
%field in computer science) is built by taking abstract ideas and inventing
%creative solution the problems in the field. In cryptography, these solutions
%take the form of alorithms and their corresponding pratical implementations.

I think the area of interaction to which cryptography most relates is
human ingenuity. This is because cryptography (and for that matter any
field in computer science) is based on inventing creative concrete
solutions to abstract problems. In cryptography, these solutions
take the form of cryptographic alorithms and their implementations
and these problemes take the form of mathematical equations.

Based on my area of interaction and my interests, human ingenuity and
learning about both applied and theoretical cryptography respectively, I
came up with the following inquiry question: "What ingenious ideas and
processes are involved in modern cryptographic theory, and how do they
function in practice?".

I decided that the best way to achieve my goal and answer my inquiry
question was to write three seperate cryptographic algorithms embodying
my research and then compare and constrast these algorithms in a research
report. This goal is particularly well suited to my needs as it balances
both the applied and the theoretical parts of cryptography, allowing me to
focus on the entire subject as a whole.

Idealy, my product should follow the following specifications.

Firstly, it should be deep and complete. In my mind this means simple,
intuitive explanations and worked examples. Around 20-30 pages would be
sufficient, perhaps.

Secondly, it should include a mix of theory and real life applications.
As my inquiry question states, this is as much about practice as it is
theory.

Thirdly, it should provide code examples. Personally, I hate reading
computer science related studies unless I can verify the code myself.
If I am going to write something, it had better have code so my
readers can check the validity of my work.

Lastly, it should look good and be entertaining to read. If I am
going to write about a topic that I spent a whole year researching,
the paper I submit should convey some of my passion for the subject
and be pleasant to look at.

% -- Selection of Sources --
\section*{Selection of Sources}
\addcontentsline{toc}{section}{Selection of Sources}

My research spanned a wide variety of subjects in both theoretical and
practical cryptography.

One of the most useful books I have used is Applied Cryptography
by Bruce Schneier \cite{schneier}. It gave me invaluable information into the
practical application of cryptography with solid definitions,
great bredth and full code examples. I was sure this book was a good
resource because the author, Bruce Schneier, is a very well known
security technologist, who currently serves on the board of directors
of the International Association of Cryptologic Research and is a member
of the Advisory Board for the Electronic Privacy Information Center while
also working at a consulting firm specializing in cryptography and computer
security. Furthermore, he has also designed one of the strongest block
ciphers currently available, called \texttt{bcrypt}. I found this book
by asking about good introductory texts to cryptography on Quora.

% -- FIXME: Footnote: bcrypt
% -- FIXME: Make sure that the footnote marks are on the same page...

The next invaluable resource I used was An Introduction to Mathematical
Cryptography by Joseph H. Silverman et ali \cite{matcrypt}. This book gave me
a complete and through understanding of the underlying mathematics behind
a wide variety of cryptographic algorithms, as well as insight into
interesting mathematical techniques which could be used to break these
algorithms. The book was initially an extremly tough read, as I was not used
to its rigorous proof based method. However, with practice and much time
spent on the wonderful examples at the end of each chapter, I was able to
use this book effectively. All of the authors of this book are proffessors
of mathematics are Brown University, and have all published numerous
research articles in their respective fields of mathematics, therefore I
beleive that this book contains trustworthy information.

Another huge resource which I took advantage of was websites such as
Khan Academy and Wikibooks. Khan Academy creates and distributes short
Youtube videos explaining topics in a wide variety of disciplines.
Wikibooks provides free, detailed texts regarding a wide variety of
subjects. Although these resources were not used when looking for
information regarding topics in cryptography (as, in the case of Wikibooks,
anyone can become an author), they were extremely useful when I wanted
to review certain topics, such as binomial coefficients or how to use
the typesetting software ({\LaTeX}), properly.

Aside from these resources, I have also referred to a many different
individual articles from various online sources, ranging from an article on
elliptic curve cryptography form the NSA to articles from Wolfram Alpha
about the applications of cellular automata in cryptography.

% -- FIXME: Footnote: NSA article, Cellular Automatata article
% -- FIXME: Footnote: Wikibooks, Khan Academy

When it came to picking my resources, I opted for quality over quantity.
Although I did not choose an enourmous selection of resources as some
students have, the resources I did choose were of excellent quality
and were ultra dense in information. For example, both of the books
mentioned above are well over 450 pages in length (501 and 758
pages respectively).

\section*{Application of Information}
\addcontentsline{toc}{section}{Application of Information}

Over the course of my project, there were a slew of skills and techniques
I developed as a result of my research, which concentrated themselves in
two main categories.

Firstly, and perhaps most importantly, I learned about the actual processes
involved in the use and the construction of cryptographic systems. I
learned about techniques such as statistical analysis of text, linear
cryptanalysis and the analysis of elliptic curve discrete logarithm based
cryptosystems, just to name a few. These techniques were applied directly
to my product in two distinct ways. Firstly, all of these ideas were explained in detail, with worked examples.
This should demonstrate my ability to apply this information through my
ability to teach it to others. Secondly, I have written programs for
both the encryption and the cracking of each of the algorithms in my
product. This should further demonstrate my understanding of the subject.

Secondly, I learned to apply many auxiliary skills to acheive my project
goals. These skills were not central to answering my inquiry question, but
rather served to enchance and improve my project as a whole. One of the
most important example of this was my development of my ability to read
and understand mathematical texts. At the beginning of the project, I had
absolutely no idea how to read mathematics, which resulted in me flipping
back between pages trying to understand what was happening during a proof
or when building on a previous topic. As the project progressed, I learned
to read carefully and use my resources properly (for example, by doing
the questions at the end of each chapter, or working all of the examples
by hand). Furthermore, apart from helping me understand the texts I was
reading, this helped me structure my writing style appropriately, so I
sound clear and professional throughout my research report. Another great
example of the skills I learned to apply are my new found {\LaTeX} typsetting
skills. Before this project began, I had not even heard of {\LaTeX},
but after doing some research on better typesetting engines (Microsoft Word
does not set mathematical text very well...) I chanced upon {\LaTeX} and took
it upon myself to learn how to use it. The wonderful Wikibook on {\LaTeX}
provided me with everything I needed to know. Using this knowledge, I
constructed not only my product, but also this essay.

I realize this may sound haughty, but I trully feel that anyone who is to
grade me on my application of information should really read my product to
truly understand the extent to which I have applied the information I
learned. This being a reflective essay, the technical content of my project
has only been vaguely mentioned (as above). However, in reality, my depth
and application of knowledge is quite quite substantial and I think this
should be taken into consideration. Here is a brief example of my
application of such technical information: Early on in the project, I was
having trouble understanding why Shank's baby step giant step collision
algorithm for the discrete logarithm problem, $g^x = h$, required lists of
size slightly larger than the size of the square root of the order of the
element $g$. It seemed absolutely strange at first, however, as I continued
through the book, I learned about the Collision Theorem and how it
(practically) guarantees that you only need to check a small multiple of
the size of the square root of the list before you find a match, because
by that point the probability of not finding a match is very very low.

\section*{Achieving the Goal}
\addcontentsline{toc}{section}{Achieving the Goal}

My goal was to answer the question "What ingenious ideas and processes
are involved in modern cryptographic theory, and how do they function
in practice?", and I think that I have succeded in acheiving that goal
with regard to my product.

% -- Include num of pages in the report here once finished

My research report meets almost all of the specifications I set out to
acheive. It is deep and complete, with lots of examples and many
simple yet detailed explanations. It focuses on both the practical
and theoretical sides of cryptography. It gives full code examples
of all of the cryptographic algorithms analyzed, as well as several
extra programs which were used in the analysis of these algorithms.
And finally, it has been beautifully set using {\LaTeX}, and includes
gorgeous tables and exquisite mathematical typesetting.

Although I cannot vouche for how pleasant my report is to read, being
the author, I do beleive it flows quite nicely, and although far from
being professionaly written, conveys my passion and interest in the
subject.

Based on all of the above, I would give myself a level 4.

% -- Does your supervisor agree?

I think the two main criteria which define the success of this project
is the meeting of the first and second specifications. The other two
are simply more personal goals which are not directly related to my
inquiry question or the topic of my project.

\section*{Reflecting on Learning}
\addcontentsline{toc}{section}{Reflecting on Learning}

Throughout my project, my learning was divided into two main
categories.

The first, and largest category, was dedicated to my learning of
cryptography. I learned about a myriad of topics ranging from
the construction and analysis of cryptographic algorithms to number
theory. During this learning, I was learned to hone my mathematical and
programmatical skills. One of the most important improvements
in my mathematical skill was my ability to read mathematical texts
properly and understand the material. At the beginning of the my project,
I was reading dense mathematics as though it were a novel. I soon realized
that truly understanding mathematical texts involves reading, re-reading,
analyzing and working through examples. I learned not only to understand
the authors conclusions, but also what lead them to those conlusions, and
how their proof is applied in the context of the problem. From this, I
learned how I should be constructing my proofs and logical reasoning, and
how to imitate the style these authors used in my own research report.

The second category was about intrapersonal learning. This project really
pushed the limits of my perseverance and my patience. Sometimes, I would
stare at my textbook for a good 45 minutes, pulling my hair out trying to
figure out why the math worked the way it did. Sometimes, I would have
to work 15 or more problems from the back of the book before I understood
the real implications of what was going on. I learned to know when I was
in the mood to understand something, and when I was wasting my time hitting
my head against the wall. I also realized the importance of occasional
breaks, despite tight deadlines, which always helped me clear my mind
and focus on the task at hand.

All of this has significantly improved my self direction as a learner.

As this project project progressed, I came to a very profound
conclusion regarding cryptography, mathematics and the nature of
human ingenuity. I realized that simply by making up structure, and
imagining new possiblities, humans have the capacity to reveal
deep truths about our universe. And in that respect, human acheivement
is only limited by out imaginations. Although this idea seems trivially
simple, it has changed the way I think about science, innovation and
especially, mathematics. I have come to seethe intrinsic beauty of
mathematical structure, and its ability to reveal amazing and wonderful
truths through the sheer power of imagination.

If I were to do a similar project in the future, I would improve my
management of time by allocating more of it towards writing the
research report and less of it towards doing research. Although
thourough research was crucial to the success of my product, I felt
as though I could progressed through my research report with less
stress had I started working on it earlier. This would have also
allowed me to have more time to edit and refine my work, which is
always appreciated. I think I could acheive this next time by allocating
work more granularly, rather than in large chunks. This way, I would
be able to stop and asses myself more often, so that I can makes
changes in my plan earlier in the process. For example, during
this project, I allocated work in month based chunks, meaning all
of my work was due at the end of the month. Rather than do this,
I should have allocated work in week based chunks, so that my
goals wouldn't seem as large and I would be able to make more
ammendments to my plan as the project progressed.

All in all, I am really glad I did this project. I have really
taken my abilities to their full potential. I learned an incredible
amount, not only about cryptography, but also about myself.

% -- FIXME: Bad conclusion, needs work

%All in all, I am really glad I did this project. Although the
%its outcome was important, I feel the real gain has been in my
%self confidence and my ability to persevere and deliver results.
%I now know myself better, and I can make more accurate judgements
%regarding my strengths and my weaknesses.

%All in all, I am really glad I did this project. Not only have
%I learned an incredible amount about cryptography, but I have
%also developed my self-awareness, perseverance and ability
%to solve challenging problems.

\nocite{*}

\addcontentsline{toc}{section}{Bibliography}
\bibliographystyle{plain}
\bibliography{doc.bib}

\end{document}

% -- TODO: Bibliography

%-- TODO: Appendix!!
