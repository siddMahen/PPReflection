\documentclass[12pt, a4paper, draft]{report}
\usepackage{palatino}
\usepackage{parskip}
\usepackage{amsmath}

\begin{document}

% -- Title --
\title{An Exploration of Cryptography}
\author{Internation School of Helsinki\\\\
    Siddharth Mahendraker\\
    \texttt{siddharth\_mahen@me.com}\\\\
    Word Count: ~820}
\date{2012}
\maketitle

% -- ToC --
\setcounter{page}{1}
\pagenumbering{roman}
\tableofcontents
\clearpage

% -- The Goal --
\section{The Goal}
\setcounter{page}{1}
\pagenumbering{arabic}

%Question: "What ingenious ideas and processes are involed in
%modern cryptographic theory, and how do they function in practice?"

I have always been interested in computer sciences. It has captivated me
since my youth, and to this day, I continue to write programs, publish
open source software, and contribute to others' work.

Therefore, when the time came to pick a topic for my personal project,
computer sciences seemed like the obvious choice.

My almost insatisfiable craving for computer science related knowledge
has led me down several very interesting paths. I have previously taken my
own time to learn about topic such as fundamental computation theory,
artificial intelligence, algorithm and data structure theory and software
engineering. However, computer science still holds many wonders for
me to discover, and cryptography was one of them.

I chose to learn about cryptography because it was a field of computer
science I genuinely knew nothing about, yet contained so many intersting
ideas. What really drew me in was the fundamental problem that cryptography
solved: it allowed people to communicate securly over completely insecure
channels. I was really interested in learning about the mathematical theory
underneath cryptography and how it could be implemented in the real world.

%I think the area of interaction to which cryptography most relates is
%human ingenuity. This is because cryptography (and for that matter any
%field in computer science) is built by taking abstract ideas and inventing
%creative solution the problems in the field. In cryptography, these solutions
%take the form of alorithms and their corresponding pratical implementations.

I think the area of interaction to which cryptography most relates is
human ingenuity. This is because cryptography (and for that matter any
field in computer science) is based on inventing creative, concrete
solutions to abstract problems. In cryptography, these solutions
take the form of cryptographic alorithms and their implementations
and these problemes take the form of mathematical equations.

Based on my area of interaction and my interests, human ingenuity and
learning about both applied and theoretical cryptography, I came up with the
following inquiry question: "What ingenious ideas and processes are involved
in modern cryptographic theory, and how do they function in practice?".

I decided that the best way to achieve my goal and answer my inquiry
question was to write three seperate cryptographic algorithms embodying
my research and then compare and constrast these algorithms in a research
report. This goal is particularly well suited to my needs as it balances
both the applied and the theoretical parts of cryptography, allowing me to
focus on the entire subject as a whole.

Idealy, the research report should follow the following specifications.

Firstly, it should be deep and complete. In my mind this means simple,
intuitive explanations and worked examples. Around 20-30 pages would be
sufficient, perhaps.

Secondly, it should include a mix of theory and real life applications.
As my inquiry question states, this is as much about practice as it is
theory.

Thirdly, it should provide code examples. Personally, I hate reading
computer science related studies unless I can verify the code myself.
If I am going to write something, it had better have code so my
readers can check the validity of my work.

Lastly, it should look good and be entertaining to read. If I am
going to write about a topic that I spent a whole year researching,
the paper I submit should convey some of my passion for the subject
and be pleasant to look at.

% -- Selection of Sources --
\section{Selection of Sources}

My research spanned a wide variety of subjects in both theoretical and
practical cryptography.

One of the most useful books I have used is Applied Cryptography
by Bruce Schneier\footnotemark. It gave me invaluable information into the
practical application of cryptography with solid definitions,
great bredth and full code examples. I was sure this book was a good
resource because the author, Bruce Schneier is a very well known
security technologist, who currently serves on the board of directors
of the International Association of Cryptologic Research and is a member
of the Advisory Board for the Electronic Privacy Information Center while
working at a consulting firm specializing in cryptography and computer
security. Furthermore, he has also designed one of the strongest block
ciphers currently available, called \texttt{bcrypt}. I found this book
by asking about good introductory texts to cryptography on Quora.

The next invaluable resource I used is An Introduction to Mathematical
Cryptography by Joseph H. Silverman et al\footnotemark. This book gave me
a complete and through understanding of the underlying mathematics behind
a wide variety of crpytographic algorithms, as well as insight into
intersting mathematical techniques which could be used to break these
algorithms. The book was initial an extremly tough read, as I was not used
to it's rigorous proof based method. However, with practice and much time
spent on the wonderful examples at the end of each chapter, I was able to
use this book effectively. All of the authors of this book are proffessors
of mathematics are Brown University, and have all published numerous
research articles in their respective fields of mathematics, therefore I
beleive that this book contains trustworthy information.

\footnotetext[1]{
Applied Cryptography - Protocols, Algorithms and Source Code in C
by Bruce Schneier
}

\footnotetext[2]{
An Introduction to Mathematical Cryptography
by Jeffery Hoffstein, Jill Pipher and Joseph H. Silverman
}

Another huge resource which I have used regularly is Khan Academy. It's
a sort of online learning resource which teaches you subjects in small
YouTube clips in less than 10 minutes. This was particularly useful if I
wanted to review something such as matrix multiplication, if I hadn't done
it in a while. Just to brush up my math skills in general.

Aside from these resources, I have also referred to a many different
individual articles from various online sources, ranging from an article on
elliptic curve cryptography form the NSA to articles from Wolfram Alpha
about the applications of cellular automata in cryptography.

When it came to picking my resources, I opted for quality over quantity.
Although I did not chose an enourmous selection of resources as some
students have, the resources I have chosen are of excellent quality and
are ultra dense in information and each entry in my bibliography has
been thoroughly vetted.

\section{Application of Information}

Because I knew absolutely nothing about cryptography before I began,
many of my first problems involved simply not understanding what was
going on. The solution, although it may seem slightly stupid, was
simply to study my resources very carefully and ensure I understood
the concepts presented.

For example, early on in the project, I was having trouble understanding
why Shank's baby step giant step collision algorithm for the discrete
logarithm problem, $g^x = h$, required lists of size slightly larger than
the size of the square root of the order of the element $g$. It seemed
absolutely strange at first, however, as I continued through the book, I
learned about the Collision Theorem and how it (practically) guarantees
that you only need to check a small multiple of the size of the square
root of the list before you find a match, because by that point the
probability of not finding a match is very very low.


Later in the project, my resources also helped me make desicions for
which there was no mathematical or logical reasoning. For example


Based on my research, I had already decided I would be investigating
symmetric and public key cryptography because it was the basis
for other more complex areas in cryptography such as visual
cryptography, quantum cryptography and cryptographic hashing, which
I personally really wanted to learn about outside the context of
this project.

However, I still had not decided which cryptographic algorithms
(hereafter referred to as ciphers) to write about.
I wanted to cover
a large spectrum of symmetric and public key cryptography to
demonstrate my knowledge of the topic, while simoultaneously

The first and most important instance where my research came into play
was when I had to choose which subtopic of cryptography to write in my
product. Cryptography is a very, very diverse field, with many highly
interesting subtopics, including several which really caught my eye
such as homomorphic cryptography, quantum cryptography and
cryptographic hashing.

After the first month of two of research, I realized something important,
which in hindsight, I really should have forseen earlier. The bredth of
the subject would be impossible to capture with my skills and my time
constraints. The problem was all of the subtopic in cryptography were

I would have to choose a particular subtopic in cryptography

My project was completed in 2 phases, research and the writing of my report.

The first phase, research, was the longest and most difficult phase. During
this time, I was constantly researching new and interesting things about
the realm of cryptography. I began at the basics, the information theoretic
foundation of cryptography, its assumptions and its underlying ideas. Then
I began researching different sub-ideas in cryptography, such as
mathematically difficult algorithms, probability theory, number theory
and crpytographic protocols. After every research session, I would consiously
not look into one topic I found very interesting so that next time I would
know right where to start off.

Soon however, I realized that the huge bredth of the subject was impossible
to capture with my skills and my time constraints. Therefore, of the research
I had done, I looked at only a small subset of that. I choose cryptographic
ciphers and their implementation because it was very interesting, it was
within my ability to program such programs and they came in a variety of
difficulties so I could really show off my skills in the product.
Furthermore, it was a subset of cryptography that was relatable to other
people, and has very strong connections to my area of interaction.

Then I began my second phase, the writing of the report. Because I had
already researched everything I needed to know, the content of the research
report came fairly naturally. What took the longest amount of time was
structuring what I knew into a text that made sense and flowed smoothly
from one idea to the next without doing any magic. That is, I wanted to
give the reader a real understanding of what is going on, rather than
just an answer. This was acheived simply by writing extensively
at every oppertunity. Slowly, I carved my final paper out of the giant
mess of attempts.

Note, however, that there are a myriad of other interesting topics in
the field of cryptography (such as cryptograhic hashing, homomorphic
cryptography, visual cryptography, quantum crpytography etc...) which
are equally interesting, but I sadly could not include in my project.

\section{Achieving the Goal}

My goal was to answer the question "What ingenious ideas and processes
are involved in modern cryptographic theory, and how do they function
in practice?", and I think that I have succeded in acheiving that goal
with regard to my product.

% -- Include num of pages in the report here once finished

My research report meets almost all of the specifications I set out to
acheive. It is deep and complete, with lots of examples and many
simple yet detailed explanations. It focuses on both the practical
and theoretical sides of cryptography. It gives full code examples
of all of the cryptographic algorithms analyzed, as well as several
extra programs which were used in the analysis of these algorithms.
And finally, it has been beautifully set using \LaTeX, and includes
gorgeous tables and exquisite mathematical typesetting.

Although I cannot vouche for how pleasant my report is to read, being
the author, I do beleive it flows quite nicely, and although far from
being professionaly written, conveys my passion and interest in the
subject.

Based on all of the above, I would give myself a level 4.

% -- Does your supervisor agree?

I think the two main criteria which define the success of this project
is the meeting of the first and second specifications. The other two
are simply more personal goals which are not directly related to my
inquiry question or the topic of my project.

\section{Reflecting on Learning}

Although I learned an incredible amount about crytography while
investigating and making my product

Throughout my project, my learning was divided into two main
categories.

The first, and largest category, was dedicated to my learning of
cryptography. I learned about a myriad of topics ranging from
the construction and analysis of cryptographic algorithms to number
theory. During this learning, I was learned to hone my mathematical and
programmatical skills. One of the most important improvements
in my mathematical skill was my ability to read mathematical texts
properly and understand the material. At the beginning of the my project,
I was reading dense mathematics as though it were a novel. I soon realized
that truly understanding mathematical texts involves reading, re-reading,
analyzing and working through examples. I learned not only to understand
the authors conclusions, but also what lead them to those conlusions, and
how their proof is applied in the context of the problem. From this, I
learned how I should be constructing my proofs and logical reasoning, and
how to imitate the style these authors used in my own research report.

The second category was about intrapersonal learning. This project really
pushed the limits of my perseverance and my patience. Sometimes, I would
stare at my textbook for a good 45 minutes, pulling my hair out trying to
figure out why the math worked the way it did. Sometimes, I would have
to work 15 or more problems from the back of the book before I understood
the real implications of what was going on. I learned to know when I was
in the mood to understand something, and when I was wasting my time hitting
my head against the wall. I also realized the importance of occasional
breaks, despite tight deadlines, which always helped me clear my mind
and focus on the task at hand.

All of this has significantly improved my self direction as a learner.

Although
I learned about fantastic and beautiful ideas such as those involved in
information theory, probability theory, crpytographic analysis of
algorithms, number theroy, the construction of cryptographically secure
algorithms and other such technical topics. I felt as though the true
learning

Reflecting on my endevours through the lens of Human Ingenuity has
provided a very interesting insight. As this project progressed, I came
to a very profound conclusion regarding cryptography, mathematics and
the nature of human innovation. I realized that simply by making up
structure, and imagining new possiblities, humans have the capacity
to reveal deeper truths about our universe. And in that respect,
human acheivement is only limited by our imaginations. And although
this idea seems trivially simple, it has truly changed the way I think
about science and innovation, and especially mathematics. I have come
to see the intrinsic beauty of mathematical structure, and its ability
to reveal amazing and wonderful things through the sheer power of
imagination. Things such as Diffie-Hellman key exchange continue to
amaze me. The very idea that such a structure preserves information
withouth any prior secrecy is astounding.

As this project project progressed, I came to a very profound
conclusion regarding cryptography, mathematics and the nature of
human ingenuity. I realized that simply by making up structure, and
imagining new possiblities, humans have the capacity to reveal
deep truths about our universe. And in that respect, human acheivement
is only limited by out imaginations. Although this idea seems trivially
simple, it has changed the way I think about science, innovation and
especially, mathematics. I have come to seethe intrinsic beauty of
mathematical structure, and its ability to reveal amazing and wonderful
truths through the sheer power of imagination.

If I were to do a similar project in the future, I would improve my
management of time by allocating more of it towards writing the
research report and less of it towards doing research. Although
thourough research was crucial to the success of my product, I felt
as though I could progressed through my research report with less
stress had I started working on it earlier. This would have also
allowed me to have more time to edit and refine my work, which is
always appreciated. I think I could acheive this next time by allocating
work more granularly, rather than in large chunks. This way, I would
be able to stop and asses myself more often, so that I can makes
changes in my plan earlier in the process. For example, during
this project, I allocated work in month based chunks, meaning all
of my work was due at the end of the month. Rather than do this,
I should have allocated work in week based chunks, so that my
goals wouldn't seem as large and I would be able to make more
ammendments to my plan as the project progressed.

All in all, I am really glad I did this project. Although the
its outcome was important, I feel the real gain has been in my
self confidence and my ability to persevere and deliver results.
I now know myself better, and I can make more accurate judgements
regarding my strengths and my weaknesses.

All in all, I am really glad I did this project. Not only have
I learned an incredible amount about cryptography, but I have
also developed my self-awareness, perseverance and ability
to solve challenging problems.

\end{document}
